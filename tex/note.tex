\documentclass[twoside]{amsart}

\usepackage[sc]{mathpazo}
\usepackage[utf8x]{inputenc}
\usepackage[T1]{fontenc}
\usepackage{microtype}
\usepackage[english]{babel}
\usepackage[hmarginratio=1:1,top=32mm,columnsep=20pt]{geometry}
\usepackage[hang, small,labelfont=bf,up,textfont=it,up]{caption}
\usepackage{booktabs}
\usepackage{enumitem}
\usepackage{amsmath}
\usepackage{abstract}
\renewcommand{\abstractnamefont}{\normalfont\bfseries}
\renewcommand{\abstracttextfont}{\normalfont\small\itshape}

\usepackage{hyperref}

\title{Stochastic discrete Solow models}
\author{St\'ephane Adjemian}

\date{\today}

\begin{document}

\maketitle

This note presents the Solow models distributed in the \verb+mod+ subfolder of this repository. All the variants share the same basic environment. Population and efficiency are exogenous stochastic processes. In models not predicting endogenous growth (in section \ref{sec:2} we consider a case with endogenous growth), demography and efficiency have deterministic and stochastic trends. Saving behaviour is exogenous. Depreciation rate is constant.

\section{Basic Solow model}
\label{sec:1}

Production at time $t$ is given by a constant return to scale technology:
\begin{equation}
  \label{eq:production}
  Y_t = F(K_{t-1}, A_t L_t)
\end{equation}
where $A_t$ is the efficiency of labour input, $L_t$ is the quantity of labour (or the population), and $K_{t-1}$ is the predetermined level of physical capital. Efficiency growth is assumed to be a first order autoregressive process. We have:
\begin{subequations}
  \label{eq:growingefficiency}
\begin{equation}
  \label{eq:efficiencygrowthfactor-1}
  X_t = \frac{A_t}{A_{t-1}}
\end{equation}
\begin{equation}
  \label{eq:efficiencygrowthfactor-2}
  \frac{X_t}{X^{\star}} = \left(\frac{X_{t-1}}{X^{\star}}\right)^{\rho_x}e^{\varepsilon_{x,t}}
\end{equation}
\end{subequations}
with $\varepsilon_{x,t}$ a Gaussian white noise with mean zero and variance $\sigma_x^2$, and $|\rho_x|<1$. Population growth is assumed to be a first order autoregressive process. We have:
\begin{subequations}
  \label{eq:growingpopulation}
\begin{equation}
  \label{eq:populationgrowthfactor-1}
  N_t = \frac{L_t}{L_{t-1}}
\end{equation}
\begin{equation}
  \label{eq:populationgrowthfactor-2}
  \frac{N_t}{N^{\star}} = \left(\frac{N_{t-1}}{N^{\star}}\right)^{\rho_n}e^{\varepsilon_{n,t}}
\end{equation}
\end{subequations}
with $\varepsilon_{n,t}$ a Gaussian white noise with mean zero and variance $\sigma_n^2$, and $|\rho_n|<1$. Saving and physical capital stock depreciation rates are assumed to be constant, so that the capital law of motion is given by:
\begin{equation}
  \label{eq:capitallawofmotion}
  K_t = (1-\delta)K_{t-1} + s Y_t
\end{equation}
In this section we simulate the model with the following production functions:\newline

\begin{itemize}
\item Cobb Douglas:
  \[
    F(K, L) = K^{\alpha}L^{1-\alpha}
  \]
  where $\alpha\in[0,1]$ is the elasticity of output with respect to capital.\newline
\item Constant Elasticity of Substitution:
  \[
    F(K, L) = \left(\alpha K^{\frac{\sigma-1}{\sigma}} + (1-\alpha) L^{\frac{\sigma-1}{\sigma}}\right)^{\frac{\sigma}{\sigma-1}}
  \]
  where $\sigma \geq 0$ is the elasticity of substitution between factors, and $\alpha\in[0,1]$ (which as not the same interpretation as in the Cobb-Douglas case).\newline
\item Complementary factors
  \[
    F(K, L) = \min (a^{-1}K, b^{-1}L)
  \]
  where $a$ and $b$ are positive real parameters. With this technology
  it takes $a$ unit of physical capital or $b$ unit of labour to
  obtain one unit of homogeneous good.\newline
\end{itemize}

\section{Solow model with endogenous growth}
\label{sec:2}

In this section we remove the exogenous sources of growth by replacing \ref{eq:growingefficiency} and \ref{eq:growingpopulation} with:
\begin{equation}
  \label{eq:stationarypopulation}
    \frac{A_t}{A^{\star}} = \left(\frac{A_{t-1}}{A^{\star}}\right)^{\rho_a}e^{\varepsilon_{a,t}}
\end{equation}
and
\begin{equation}
  \label{eq:stationarypopulation}
    \frac{L_t}{L^{\star}} = \left(\frac{L_{t-1}}{L^{\star}}\right)^{\rho_l}e^{\varepsilon_{l,t}}
\end{equation}
Provided that the elasticity of substitution between factors is higher
than in the Cobb-Douglas case, the model predicts endogenous growth if
the saving rate is important enough, i.e. if the following inequality
holds:
\[
s > \delta\alpha^{-\frac{\sigma}{\sigma-1}}
\]


\end{document}

%%% Local Variables:
%%% mode: latex
%%% TeX-master: t
%%% End:
